\documentclass[12pt,a4paper]{report}

\usepackage[margin=1in]{geometry}
\usepackage{lipsum}
\usepackage{times}
\usepackage{listings}
%\usepackage{hyperref}

\makeatletter
\renewcommand{\@makechapterhead}[1]{%
  {\noindent\raggedright\normalfont% Alignment and font reset
   \huge\bfseries \@chapapp\space\thechapter~~#1\par\nobreak}% Formatting
  \vspace{\baselineskip}% ...just a little space
}
\makeatother


\usepackage[colorlinks = true,
            linkcolor = blue,
            urlcolor  = blue,
            citecolor = blue,
            anchorcolor = blue]{hyperref}

\newcommand{\changeurlcolor}[1]{\hypersetup{urlcolor=#1}}  


\setcounter{secnumdepth}{3}
\setcounter{tocdepth}{3}
\renewcommand{\contentsname}{Table of Contents}

\usepackage{titletoc}

\newcommand{\setupname}[1][\chaptername]{
\titlecontents{chapter}[0pt]{\vspace{1ex}}{\bfseries#1~\thecontentslabel:\quad}{\bfseries}{\bfseries\hfill\contentspage}[]
}

\author{Anthony McGlone}\title{Compile and boot a Linux Kernel}
\begin{document}\maketitle

\tableofcontents

\setupname
\chapter{Introduction}

\section{Installing Oracle VirtualBox}
\subsection{MacOS}

1. Open a terminal. Then install Homebrew (steps located: \href{https://brew.sh/}{here})
\newline
2. Install Git by running \texttt{brew install git}
\newline
3. Open a terminal and run \texttt{git --version} (if Git is installed, the version number should be printed to the console) 

\subsection{Windows}

1. Install Git by downloading the Windows binary from git-scm (\href{https://git-scm.com/downloads}{here})
\newline
2. Run the installer
\newline
3. Open a Git Bash terminal and run \texttt{git --version} (if Git is installed, the version number should be printed to the console)


\subsection{Linux (Ubuntu)}

1. Install Git by opening a terminal and running \texttt{sudo apt-get install git}
\newline
2. Run \texttt{git --version} (if Git is installed, the version number should be printed to the console)


\subsection{Linux (Red Hat)}


1. Install git by opening a terminal and running \texttt{sudo yum install git} 
\newline
2. Run \texttt{git --version} (if Git is installed, the version number should be printed to the console)










\end{document}



















