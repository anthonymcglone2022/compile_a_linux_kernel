\documentclass[12pt,a4paper]{report}

\usepackage[margin=1in]{geometry}
\usepackage{lipsum}
\usepackage{times}
\usepackage{listings}
%\usepackage{hyperref}

\makeatletter
\renewcommand{\@makechapterhead}[1]{%
  {\noindent\raggedright\normalfont% Alignment and font reset
   \huge\bfseries \@chapapp\space\thechapter~~#1\par\nobreak}% Formatting
  \vspace{\baselineskip}% ...just a little space
}
\makeatother


\usepackage[colorlinks = true,
            linkcolor = blue,
            urlcolor  = blue,
            citecolor = blue,
            anchorcolor = blue]{hyperref}

\newcommand{\changeurlcolor}[1]{\hypersetup{urlcolor=#1}}  


\setcounter{secnumdepth}{3}
\setcounter{tocdepth}{3}
\renewcommand{\contentsname}{Table of Contents}

\usepackage{titletoc}

\newcommand{\setupname}[1][\chaptername]{
\titlecontents{chapter}[0pt]{\vspace{1ex}}{\bfseries#1~\thecontentslabel:\quad}{\bfseries}{\bfseries\hfill\contentspage}[]
}

\author{Anthony McGlone}\title{Compile and boot a Linux Kernel}
\begin{document}\maketitle

\tableofcontents

\setupname
\chapter{Introduction}

If you already have a Linux operating system (such as Ubuntu or CentOS/ Red Hat), you can skip ahead to the kernel installation instructions. 
\newline
\newline
If you are on Windows or Mac, you can use Oracle's VirtualBox to test an installation.

\section{Installing Oracle VirtualBox}

1. Read the \href{https://www.virtualbox.org/manual/ch02.html}{installation instructions} for your OS.
\newline
2. Download the VirtualBox \href{https://www.virtualbox.org/wiki/Downloads}{binary / package} for your OS.
\newline
3. Download an ISO of a Linux OS (e.g. \href{https://ubuntu.com/download/desktop}{Ubuntu}, \href{https://www.centos.org/download/}{CentOS}, \href{https://getfedora.org/workstation/download/}{Fedora}).
\newline
4. \href{https://docs.oracle.com/cd/E26217_01/E26796/html/qs-create-vm.html}{Install} the ISO / installation media.










\end{document}



















